% !TeX root=../main.tex

\chapter{مقدمه}
% دستور زیر باعث عدم‌نمایش شماره صفحه در اولین صفحهٔ این فصل می‌شود.
%\thispagestyle{empty}
\section{انگیزش}

مجازی‌سازی شبکه در چندین سال اخیر از تکنولوژی‌های مهم در ایجاد \gls{انعطاف‌پذیری} و \gls{انزوا}ی مورد نیاز برای استقرار نسل جدید برنامه‌های شبکه‌های کامپیوتری را فراهم می‌کند. بخش اساسی این فناوری مسئله تعبیه شبکه‌های مجازی درخواستی
\lr{(VN)}
 در یک شبکه فیزیکی 
 \lr{(PN)}
 می‌باشد. در ساده‌ترین شکل می‌توان این مسئله را نگاشت شبکه‌های مجازی با درخواست منابع مشخص روی شبکه فیزیکی با منابع محدود تعبیر کرد. استراتژی مورد استفاده در این مسئله به صورت محسوسی بر بهره‌برداری منابع تاثیر دارد و به تبع آن مقدار هزینه و سود تولیدی شبکه را مشخص می‌کند. با توجه به اینکه این مسئله یک مسئله‌ی 
 \lr{NP-hard}
 می‌باشد 
 \cite{vne_nphard}
	طرح یک الگوریم مناسب مد نظر محققان در ادوار مختلف بوده‌است. 
	\cite{survey_1}
	
با این حال، در مواجهه با گسترش روزافزون شبکه‌های کامپیوتری، بیشتر مطالعات فعلی از مشکلات مقیاس‌پذیری رنج می‌برند که استفاده عملی آنها را در محیط‌های حساس مدرن محدود می‌کند. به‌علاوه، با بزرگتر شدن این شبکه‌ها، کیفیت راه‌حل‌های موجود به شدت کاهش ‌می‌یابد.
یک روش امیدوار کننده برای کاهش این مسائل استفاده از مرحله پیش پردازش برای کمک به الگوریتم تعبیه شده  می‌باشد. الگوریتم‌ها با استفاده از این مرحله کار خود را سریعتر و با موفقیت بالاتر انجام می‌دهند. 
به همین ترتیب، برخی از رویکردها استراتژی‌های مختلفی را بررسی کرده‌اند. از این استراتژی‌ها می‌توان محدودکردن گزینه‌های جاسازی به زیرمجموعه ای از گره‌های فیزیکی 
\cite {neurovine} 
، کوچک کردن اندازه شبکه‌های مجازی
\cite {Wang_2019_Globecom} 
و رد شبکه‌های مجازی  که به سختی نگاشت می‌شوند بدون اجرای الگوریتم جاسازی واقعی
 \cite {Blenk_2016_CNSM}
  را نام برد.
  
  اگرچه پیشنهادهای فعلی اثربخشی تکنیک پیش پردازش را نشان داده اند، اکثر آن‌ها در بررسی جنبه‌های اساسی مسئله نگاشت شبکه مجازی (برای مثال توپولوژی شبکه و منابع متعدد) برای حفظ هزینه‌های محاسباتی خود در یک سطح قابل تحمل‌، کوتاهی می‌کنند.
  آن‌ها همچنین به‌طور معمول بر اساس الگوی سنتی مدل‌محور طراحی ‌می‌شوند که ن‌می‌تواند الگوهای اساسی  شبکه و اتصالات پنهانی را که معمولاً مختص سیستم هدف هستند‌، در نظر بگیرند.
  رویکردهای داده‌محور ‌می‌توانند با استفاده از معماری یادگیری ماشین (به عنوان مثال شبکه‌های عصبی عمیق کانولوشن) این الگوها را شناسایی و بهره‌برداری کنند، با این حال، ماهیت غیراقلیدسی ساختارهای شبکه (گراف‌های وزن‌دار با گره‌های دارای ویژگی)
\cite {graphNN_RL} 
  مسئله و روند طراحی را پیچیده می‌کند.
  
  \gls{gnn}
  \cite {graph_nn}
  یک معماری جدید برای یادگیری ماشین است که ‌می‌تواند همزمان با تلفیق توپولوژی، چندین ویژگی را در گره‌های نمودار یاد‌بگیرد.
%  
  به طور خاص، یک \gls{gnn} ‌می‌تواند به طور خودکار نمایش متراکم هر گره را در شبکه بیاموزد که شامل اطلاعات مربوط به گره، همسایگان آن (تا فاصله مورد نظر) و توپولوژی متصل کننده آنها باشد.
%
  با استفاده از این چارچوب در PN، 
  \gls{خوشه‌بندی}
   گره‌های فیزیکی (یعنی سرورها) بر اساس ظرفیت منابع آنها (به عنوان مثال
   \lr{CPU}
     و
   \lr{RAM}
      ) و موقعیتشان در توپولوژیکی امکان پذیر است.
%
  سپس، خوشه‌های محاسبه شده ‌می‌توانند با حذف در نظر گرفتن گزینه‌های مشابه و فراهم آوردن فرصت تمرکز روی بررسی گزینه‌های مجزا، فضای جستجو را کوچک کنند.
  
  هدف ما در این پروژه بهره‌برداری از \gls{spgnn} است
   \cite {graphSage,argva} 
، که عملیات موازی‌سازی را فراهم ‌می‌کند. این روش  با در نظرگرفتن منابع سرور‌ها و توپولوژی شبکه، فضای جستجوی مسئله نگاشت شبکه مجازی را به صورت کارآمد و معنی‌دار انجام می‌دهد.
 
\section{الگوریتم پیشنهاد شده}

ما یک الگوریتم VNE ارائه ‌می‌دهیم که از \gls{gnn} برای سرعت بخشیدن و افزایش کارایی استفاده ‌می‌کند.
%
به طور خاص، ما یک \gls{argva} برای خوشه بندی سرورهای فیزیکی بر اساس ظرفیت منابع و وضعیت شبکه طراحی ‌می‌کنیم.
%
الگوریتم نگاشت، سپس از این خوشه‌های محاسبه شده برای یافتن و بررسی کارآمد سرورهایی استفاده می‌کند که مجموعه‌ای متنوع از گزینه‌های جاسازی را ارائه ‌می‌دهند.
%
ما به جای استفاده از \lr{GNN}های مبتنی بر طیف
 \cite {graphNN_RL}
که یکباره کل نمودار را پردازش ‌می‌کنند، از \gls{spgnn} استفاده ‌می‌کنیم که ‌می‌توانند در چند گره  محدود کار کنند.
%
در نتیجه، مدل ما می‌تواند از پردازش این دسته از گره‌ها به طور موازی استفاده کند، که مقیاس پذیری را به طور قابل توجهی بهبود می‌بخشد.
%
علاوه بر این، رویکردهای مبتنی بر طیفی به یک فوریه گراف تکیه می‌کنند و یک گراف ثابت را  در تمام مراحل در نظر می‌گیرند، که تعمیم و انتقال پذیری آنها را به شدت محدود ‌می‌کند. \lr{GNN}های فضایی با انجام عمل کانولوشن خود به صورت محلی روی هر گره، این محدودیت را کاهش ‌می‌دهند. بنابراین، در محیط‌های پویا که ظرفیت شبکه پس از هر ورود یا خروج شبکه مجازی تغییر ‌می‌کند، \gls{spgnn} به واسطه‌ی عملیات محلی، تعمیم پذیری، مقیاس پذیری و در نتیجه عملکرد بهتری را ارائه ‌می‌دهند.

مشارکت‌های اصلی ما به شرح زیر خلاصه ‌می‌شود:
\begin{itemize}
	\item 
	ما یک شبکه فیزیکی با منابع چند بعدی (به عنوان مثال \lr{CPU} و \lr{GPU}) را به عنوان یک گراف دارای \emph{ویژگی‌های گره} مدل ‌می‌کنیم.
	\item 
	ما در یک  اتوانکودر متخامصانه گرافی \gls{spgnn} را به کار بگیریم، که ‌می‌تواند با جمع آوری اطلاعات همسایه، عملیات کانولوشن را در گراف را انجام دهد. در نتیجه، روش ما سطح بالاتری از قابلیت موازی سازی و تعمیم پذیری را فراهم می‌کند.
	\item 
	ما یک تابع مناسب را تعریف ‌می‌کنیم که توسط \lr{GNN} برای جمع‌آوری اطلاعات مربوط به هر سرور فیزیکی، همسایگان آن و شبکه اتصال  آنها (یعنی  توپولوژی و پهنای باند) استفاده ‌می‌شود. ما از قطر مورد انتظار شبکه‌های مجازی برای تعیین عمق تجمیع اطلاعات در شبکه‌های گرافی فضایی استفاده ‌می‌کنیم.
	\item 
	ما از اطلاعات جمع شده برای خوشه‌بندی سرورها بر اساس قابلیت نگاشت‌ آن‌ها و یافتن مجموعه متنوعی از سرورها برای شروع فرآیند نگاشت شبکه‌ی مجازی استفاده ‌می‌کنیم. برای تعیین تعداد خوشه‌ها نیز از روش  
	\lr{elbow}
	\cite{elbowMethod} 
	استفاده ‌می‌کنیم.
	\item 
	ما الگوریتم خود را شبیه سازی ‌می‌کنیم تا سرعت و عملکرد آن را نشان دهیم و آن را با الگوریتم‌های اخیر \lr{VNE} مقایسه کنیم.
	
\end{itemize}
