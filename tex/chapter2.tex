% !TeX root=../main.tex
\chapter{مروری بر مطالعات انجام شده}
%\thispagestyle{empty} 
ما کارهای مرتبط را در سه گروه مرور می‌کنیم. برای یک بررسی جامع اخیر به
 \cite {survey_1} 
 مراجعه کنید.
 
 \section{
الگوریتم‌های عمومی \lr{VNE} 
}
برای رسیدگی به اندازه و پیچیدگی مسئله ، نویسندگان در 
\cite {Kibalya_2020_CN}
 از اصول برنامه نویسی پویا برای تجزیه شبکه‌های مجازی به مجموعه‌ای از بخش‌ مسیرهای جدا از لبه 
 \LTRfootnote{edge-disjoint}
 استفاده کرده و از یک تغییر شکل چند لایه گرافی برای تعبیه شبکه مجازی در بخش‌های بدست آمده استفاده کردند. 
 %
نویسندگان در
 \cite {Cao_2020_WCNCW} 
 یک تنظیمات چند بعدی را در نظر گرفتند،که یک ویژگی امنیتی با هر گره و پیوند فیزیکی مرتبط است. آنها جاسازی حریصانه ای را برای تخصیص منابع انجام دادند در حالی که احتمال حملات مخرب با هدف شبکه‌های مجازی را به حداقل می‌رساند.
 %
 دو بعد منبع (
 \lr{CPU} و \lr{RAM}
 )
 در 
 \cite {Pentelas_2020_NOMS} 
 در نظر گرفته شده است. نویسندگان از نسبت
 $\frac{\text{CPU}}{\text{RAM}}$
 برای به دست آوردن معیاری برای رتبه‌بندی گره‌ها استفاده می‌کنند تا عمل نگاشت را انجام دهند. 
 نویسندگان در
  \cite {Hosseini_2019_TNSM} 
  با مدلسازی تأخیر در پیوندهای فیزیکی به عنوان متغیرهای تصادفی با میانگین و واریانس شناخته شده، تأخیر پیوندهای مجازی را در نظر می‌گیرند.
  هیچ یک از این آثار نمی‌توانند از تکنیک‌های یادگیری ماشین برای بهره‌برداری از داده‌های عملیاتی موجود در سیستم هدف استفاده کنند.
  
\section{
الگوریتم‌های \lr{VNE} مبتنی بر یادگیری
}
 
 نویسندگان در 
 \cite {graphNN_RL}
  از الگوریتم 
  \lr{actor-critic}
   استفاده کردند تا نگاشت را از طریق تکنیک اکتشاف و بهره برداری به صورت خودکار انجام دهد. در این الگوریتم از یک شبکه عصبی گرافی کانولوشن مبتنی بر طیف برای استخراج ویژگی‌های شبکه فیزیکی استفاده می‌شود که محیط را مدل می‌کند.
   روشی به نام \lr{DeepViNE} در 
   \cite {Dolati_2019_Infocom} 
   ارائه شده است که شبکه‌‌های فیزیکی و مجازی را به صورت تصاویری با چند لایه رمزگذاری می‌کند. سپس تصاویر را به یک عامل یادگیری تقویتی عمیق
   \LTRfootnote{deep reinforcement learning}
    با استفاده از لایه‌های کانولوشن انتقال می‌دهد تا استراتژی نگاشت را به حداکثر برساند. \lr{DeepViNE} محدود به شبکه‌هایی با توپولوژی \lr{grid} است.
    هیچ یک از این کارها مکانیزمی‌ برای مشکل مقیاس‌پذیری در این مسئله ارائه نمی‌دهند.
    
    \section{پیش‌پردازش برای \lr{VNE}}
 روش پیشنهادی در
  \cite {neurovine}
   از اندازه‌ی گراف شبکه مجازی برای تعیین تعداد مناسب گره‌های فیزیکی برای نگاشت استفاده می‌کند. سپس از یک شبکه Hopfield (نوعی شبکه عصبی مصنوعی مکرر) برای انتخاب زیر‌مجموعه‌ای از گره‌ها با اندازه تعیین شده استفاده می‌کند که احتمال نگاشت موفقیت آمیز را به حداکثر می‌رساند.
   نویسندگان در
  \cite {Blenk_2016_CNSM}
   از شبکه‌‌های عصبی مکرر برای طراحی مکانیزم کنترل پذیرش استفاده کردند که پیش بینی می‌کند آیا منابع فیزیکی موجود برای نگاشت درخواست‌های مجازی جدید کافی است یا نه. رد زودهنگام درخواست‌های غیرقابل تعبیه، اجرای ناموفق الگوریتم نگاشت را از بین می‌برد و در نتیجه زمان پاسخ را کاهش می‌دهد.
   یک الگوریتم پارتیشن‌بندی شبکه در
    \cite {Wang_2019_Globecom} 
    استفاده می‌شود تا از طریق تقسیم گره‌های مجازی به تعداد خاصی از گروه‌ها (براساس گره فیزیکی و منابع پیوندها) که دارای اتصالات سبک هستند ، از اندازه شبکه‌‌های مجازی بکاهد.
    نویسندگان در 
    \cite {He_2020_CN}
     تئوری میدان
     \LTRfootnote{field theory}
      را برای استخراج ویژگی‌های شبکه فیزیکی اعمال کردند. سپس خوشه‌بندی طیفی انجام دادند تا گره‌های دارای شباهت زیاد در منابع و اتصالات را در مناطقی جمع کنند که ظرفیت کافی برای فرآیند نگاشت را فراهم می‌کند.
      هیچ یک از این کارها از مکانیزم سیستماتیک مانند \lr{GNN} برای در نظر گرفتن همزمان سرورها با منابع چند‌بعدی و توپولوژی شبکه استفاده نمی‌کنند.

% 
% 
% \newpage
% 
%
%\section{مقدمه}
%هدف از این فصل که با عنوان‌های  «مروری بر ادبیات موضوع%
%\LTRfootnote{Literature Review}»،
%«مروری بر منابع» و یا «مروری بر پیشینه تحقیق%
%\LTRfootnote{Background Research}»
%معرفی می‌شود، بررسی و طبقه‌بندی یافته‌های تحقیقات دیگر محققان در سطح دنیا و تعیین و شناسایی خلأهای تحقیقاتی است. آنچه را که تحقیق شما به دانش موجود اضافه می‌کند، مشخص کنید. طرح پیشینه تحقیق%
%\LTRfootnote{Background Information}
%یک مرور محققانه است و تا آنجا باید پیش برود که پیش‌زمینهٔ تاریخی مناسبی از تحقیق را بیان کند و جایگاه تحقیق فعلی را در میان آثار پیشین نشان دهد. برای این منظور منابع مرتبط با تحقیق را بررسی کنید، البته نه آنچنان گسترده که کل پیشینه تاریخی بحث را در برگیرد. برای نوشتن این بخش:
%\begin{itemize}
%	\item
%	دانستنی‌های موجود و پیش‌زمینهٔ تاریخی و وضعیت کنونی موضوع را چنان بیان کنید که خواننده بدون مراجعه به منابع پیشین، نتایج حاصل از مطالعات قبلی را درک و ارزیابی کند.
%	\item
%	نشان دهید که بر موضوع احاطه دارید. پرسش تحقیق را همراه بحث و جدل‌ها و مسائل مطرح شده بیان کنید و مهم‌ترین تحقیق‌های انجام شده در این زمینه را معرفی نمائید.
%	\item
%	ابتدا مطالب عمومی‌تر و سپس پژوهش‌های مشابه با کار خود را معرفی کرده و نشان دهید که تحقیق شما از چه جنبه‌ای با کار دیگران تشابه یا تفاوت دارد.
%	\item
%	اگر کارهای قبلی را خلاصه کرده‌اید، از پرداختن به جزئیات غیرضروری بپرهیزید. در عوض، بر یافته‌ها و مسائل روش‌شناختی مرتبط و نتایج اصلی تأکید کنید و اگر بررسی‌ها و منابع مروری عمومی دربارهٔ موضوع موجود است، خواننده را به آنها ارجاع دهید.
%\end{itemize}
%
%\section{تعاریف، اصول و مبانی نظری}
%این قسمت ارائهٔ خلاصه‌ای از دانش کلاسیک موضوع است. این بخش الزامی نیست و بستگی به نظر استاد راهنما دارد.
%
%\section{مروری بر ادبیات موضوع}
%در این قسمت باید به کارهای مشابه دیگران در گذشته اشاره کرد و وزن بیشتر این قسمت بهتر است به مقالات ژورنالی سال‌های اخیر (۲ تا ۳ سال) تخصیص داده شود. به نتایج کارهای دیگران با ذکر دقیق مراجع باید اشاره شده و جایگاه و تفاوت تحقیق شما نیز با کارهای دیگران مشخص شود. استفاده از مقالات ژورنال‌های معتبر در دو یا سه سال اخیر، می‌تواند به اعتبار کار شما بیافزاید.
%
%\section{نتیجه‌گیری}
%‌در نتیجه‌گیری آخر این فصل، با توجه به بررسی انجام شده بر روی مراجع تحقیق، بخش‌های قابل گسترش و تحقیق در آن حیطه و چشم‌اندازهای تحقیق مورد بررسی قرار می‌گیرند.	در برخی از تحقیقات، نتیجه نهایی فصل روش تحقیق، ارائهٔ یک چارچوب کار تحقیقی 
%\lr{(research framework)}
%است.