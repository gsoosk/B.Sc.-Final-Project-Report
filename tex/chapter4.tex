% !TeX root=../main.tex
\chapter{نتایج}
\label{section_results}

ما شبیه سازی‌های گسترده ای را انجام داده‌ایم تا عملکرد روش پیشنهادی خود را از نظر نسبت پذیرش \lr{VN}‌، درآمد‌، هزینه و \gls{utilization} در برابر الگوریتم‌های دیگر نشان دهیم.
ما برای ساخت شبکه‌های فیزیکی و مجازی از مدل گراف‌های تصادفی
 \cite {randomGraph}
  استفاده می‌کنیم.
همه الگوریتم‌ها در پایتون \lr{$3.6.9$} اجرا می‌شوند. محاسبات در رایانه ای با پردازنده
\lr{Intel$^{\tiny{\textregistered}}$ Core\texttrademark\,\,$i5-9500T$}
 در فرکانس 
 \lr{$2.2 - 3.7$~GHz}
 و مقدار حافظه موقت 
 \lr{$8$~GB}
 اجرا‌ شده‌اند. 
 برای بررسی قابلیت موازی سازی الگوریتم خود ما از 
 \lr{Google Colaboratory GPU notebook~\cite{colab}}
 استفاده کرده‌ایم که از دو پردازنده 
 \lr{Intel$^{\tiny{\textregistered}}$ Xeon$^{\tiny{\textregistered}}$}
 در فرکانس 
 \lr{2.0 GHz}
 با مقدار حافظه 
 \lr{13 GB}
 و پردازنده گرافیکی 
\lr{Nvidia$^{\tiny{\textregistered}}$ Tesla$^{\tiny{\textregistered}}$ P$4$} 
بهره می‌برد. 

\section{پارامتر‌های شبیه‌سازی}

در این بخش‌، پارامترهایی را که در طول ارزیابی استفاده کردیم‌، تعریف و توضیح می‌دهیم. جدول 
\ref {tab:simulation_parameters}
 پارامترهای مهم را خلاصه می‌کند.

 \begin{table}[t]
     \centering
     \caption{پارامتر‌های شبیه‌سازی}
     \label{tab:simulation_parameters}
     \begin{tabular}{rc}
     \toprule
         پارامتر & مقدار \\
     \midrule
        تعداد گره‌های شبکه فیزیکی & $100$\\
        تعداد گره‌های یک شبکه مجازی & $4-10$ \\
         نرخ وجود لینک‌های شبکه فیزیکی &$0.4$ \\
         نرخ وجود لینک در شبکه‌های مجازی & $0.7$ \\
         عمر شبکه‌های مجازی & $100-900$ واحد زمانی \\
         مقدار پهنای باند مورد نیاز پیوند‌ها در شبکه مجازی & $4 - 10$ \\
         تعداد هسته \lr{CPU} درخواستی در شبکه مجازی & $ 4 - 10$ \\
         تعداد هسته \lr{GPU} درخواستی در شبکه مجازی & $3 - 9$  \\
     \bottomrule
     \end{tabular}
 \end{table}

\subsection{شبکه فیزیکی}
برای مدل سازی شبکه فیزیکی، ما یک پیکربندی معمولی در دنیای واقعی را در نظر می‌گیریم که از مقدار 1.2 حافظه و صد واحد پردازشی و هسته گرافیکی تشکیل شده است.
 مشابه 
\lr{Dell\texttrademark\ PowerEdge\texttrademark\ R$910$ } و \lr{Intel$^{\tiny{\textregistered}}$ Xeon$^{\tiny{\textregistered}}$}
که دارای دارای پردازنده‌های مقایس‌پذیر پس از 
\lr{Hyperthreading}
 و  پردازنده گرافیکی 
 \lr{Nvidia$^{\tiny{\textregistered}}$ GeForce$^{\tiny{\textregistered}}$ GT $330$}
 که در دیتاسنتر‌های واقعی مورد استفاده هستند. 
  سپس‌، ما توزیع نرمال را رو این مقادیر برای ایجاد تنوع و ناهمگنی اعمال می‌کنیم. در نتیجه‌، ما شبکه ای با سرورهای 100تایی را در نظر می‌گیریم که احتمال $40\%$  ارتباط فیزیکی مستقیم بین هر جفت سرور وجود را دارد. هر پیوند فیزیکی با پهنای باند آن (در \lr{Mbps}) مشخص می‌شود که به طور تصادفی از توزیع‌های نرمال $\mc{N}(100, 400)$ نتخاب می‌شود. هر گره فیزیکی با قدرت پردازنده (تعداد هسته)‌، ظرفیت حافظه (در گیگابایت) و توان GPU (تعداد هسته) مشخص می‌شود. این مقادیر به ترتیب از توزیع‌های نرمال $\mc{N}(100, 400)$‌، $\mc{N}(1200, 300)$ و $\mc{N}(100, 400)$ انتخاب می‌شوند.
  
  \subsection{شبکه مجازی}
  	تعداد گره‌های مجازی در هر \lr{VNR} به طور تصادفی از فاصله $ [4‌، 10] $ انتخاب می‌شود. احتمال وجود پیوند مجازی بین دو گره مجازی $0.7$ است. تقاضای پهنای باند پیوندهای مجازی توزیع نرمال $\mc{N}(10, 4)$ را دنبال می‌کند. تقاضاهای مقدار پردازنده‌، حافظه و پردازنده گرافیکی هر گره مجازی به ترتیب از توزیع‌های نرمال $\mc{N}(10, 4)$‌، $\mc{N}(30, 9)$ و $\mc{N}(10, 4)$ حاصل می‌شود. هر \lr{VN} یک طول عمر دارد که به طور تصادفی از توزیع نرمال $\mc{N}(100, 900)$ انتخاب می‌شود. \lr{VN}‌ها با توجه به میزان نرخ ورود می‌رسند و در طول عمر خود در شبکه فیزیکی باقی می‌مانند. نرخ ورود VN روی $ 2$ در هر واحد زمان تنظیم شده و شبیه سازی در $ 2000 $ واحد زمان انجام می‌شود.
  	
  	\subsection{الگوریتم‌های مورد مقایسه}
  	علاوه بر  \ourAlg\ ، الگوریتم‌های زیر را برای مقایسه پیاده سازی کرده ایم.
  	
  \begin{itemize}
  	\item 
  	\lr{\textbf {FirstFit}}:
  	الگوریتمی ‌است که گره‌های مجازی را در اولین گره فیزیکی با ظرفیت کافی جاسازی می‌کند.
  	\item 
  	\lr{\textbf{BestFit}}:
  	الگوریتمی که گره فیزیکی را با حداکثر ظرفیت پردازنده انتخاب کرده و آن را با منابع درخواستی گره مجازی پر می‌کند.
  	\item 
  	\lr{\textbf{GRC~\cite{grc}}}:
  	یک الگوریتم مبتنی بر رتبه بندی گره.
  	\item 
  	\lr{\textbf{NeuroViNE}}:
  	الگوریتمی مبتنی بر مکانیزم کاهش فضای جستجو. این الگوریتم زیرگراف‌های مربوطه را توسط یک شبکه \lr{Hopefield} استخراج می‌کند. سپس‌، از \lr{GRC} برای جاسازی \lr{VN} در زیرگراف‌های نامزد استفاده می‌کند.
  \end{itemize}

  	از آنجا که \lr{GRC} و \lr{NeuroViNE} فقط می‌توانند سرور با یک منبع را در نظر بگیرند (یعنی \lr{CPU})‌، ما از آنها برای ارزیابی مقیاس پذیری و نگاشت شبکه‌ها مجازی ب یک منبع استفاده می‌کنیم. سپس‌، در یک معیار جداگانه‌، از تغییرات دیگری از الگوریتم‌های \lr{Best Fit} و \lr{First Fit} برای بررسی تنظیمات عمومی با چندین منبع در گره‌ها استفاده می‌کنیم.
  	
  	پارامتر‌های $\alpha$، $\beta$‌، $\kappa_1$ و $\kappa_2$ به ترتیب به مقادیر 30‌، 3‌، 10 و 10 تنظیم شده‌اند. 
  	\section{معیار‌ها}
  	\subsection{موازی‌سازی}
  	
  	\begin{figure}
  		\centering
  		\begin{minipage}[t]{.48\linewidth}
  			\centering
  			\resizebox{\linewidth}{!}{%
  				\input{plots/clusteringTimes}
  			}%
  			\caption{
  			مقایسه زمان اجرای \lr{GPU} و \lr{CPU}
  		}
  			\label{fig:gpu-cpu-comparision}
  		\end{minipage}\hfill
  		\begin{minipage}[t]{.48\linewidth}
  			\centering
  			\resizebox{\linewidth}{!}{%
  				\input{plots/acceptanceRatios/load1000-maxLink50}
  			}
  			\caption{
  			مقایسه نرخ پذیرش شبکه مجازی
  		}
  			\label{fig:ar1}
  		\end{minipage}
  	\end{figure}
  
  در ابتدا‌، قابلیت موازی سازی \ourAlg\ را که توسط معماری فضایی \lr{GNN} ارائه شده و عملیات محلی گره‌های منفرد  انجام می‌شود را بررسی می‌کنیم. برای این منظور‌، ما اندازه شبکه فیزیکی را از ۱۰۰ سرور به $ 1500 $ سرور افزایش دادیم و مدت زمان اجرای الگوریتم را روی \lr{CPU} و \lr{GPU} اندازه گیری کردیم. شکل
  \ref{fig:gpu-cpu-comparision}
  تفاوت قابل توجهی را بین نتایج پردازنده و پردازنده گرافیکی نشان می‌دهد‌، به طوری که زمان اجرای قبلی به طور تصاعدی رشد می‌کند در حالی که زمان اجرا  روی پردازنده گرافیکی نسبتاً ثابت است. به طور خاص‌، این موازی سازی باعث شده است که زمان اجرا به طور متوسط ​​حدود 8 برابر و در بهترین حالت 20 برابر افزایش یابد. علاوه بر این‌، ما حساسیت \ourAlg\ به اندازه شبکه مجازی را با افزایش آن تا $50\%$ شبکه فیزیکی (یعنی ۵۰ گره) بررسی کردیم و فقط 3 ثانیه افزایش در زمان عملیات نگشات مشاهده کردیم. 
  بنابراین‌، نتیجه می‌گیریم که اندازه شبکه فیزیکی عامل اصلی محدود کردن مقیاس پذیری است. شایان ذکر است که نه \lr{GRC} و نه \lr{NeuroViNE} مکانیزمی را برای این کار فراهم نمی کنند و از قابلیت پردازش موازی موجود بهره نمی‌گیرند. 
  
  \subsection{نرخ پذیرش}
  
  نرخ پذیرش شبکه مجازی در  طولانی مدت معیار مهمی است که بر سود سیستم تأثیر به‌سزایی می‌گذارد. شکل 
  \ref {fig:ar1}
   نرخ پذیرش الگوریتم‌های مختلف را در یک شبیه سازی طولانی در 2000  واحد زمانی را نشان می‌دهد. توجه داشته باشید که در این دوره‌، نسبت پذیرش به دلیل ثابت بودن روند ورود شبکه مجازی‌، به حالت پایدار می‌رسند و ثابت می‌مانند. الگوریتم \lr{First Fit } منابع را بیش از حد تکه تکه می‌کند و افت قابل توجهی در ظرفیت پذیرش شبکه‌های مجازی جدید را تجربه می‌کند. \ourAlg\ در مقایسه با الگوریتم‌های \lr{NeuroViNE}‌، \lr{GRC} و \lr{First Fit } به ترتیب حدود$20\%$، $25\% $ و $ 100\% $ نرخ پذیرش را  افزایش می‌دهد. عملکرد \lr{GRC} و \lr{Best Fit} مشابه هستند.
   
   
   \begin{figure}[t]
   	\centering
   	% \vspace{0.05in}
   	\begin{minipage}{0.43\linewidth}
   		\centering
   		\resizebox{\linewidth}{!}{
   			\revenueFig{load1000-maxLink10}
   		}
   		\caption{درآمد}
   		\label{fig:rev-load1000}
   	\end{minipage}
   	\hfil
   	\begin{minipage}{0.43\linewidth}
   		\centering
   		\resizebox{\linewidth}{!}{
   			\costFig{load1000-maxLink10}
   		}
   		\caption{هزینه}
   		\label{fig:cost-load1000}
   	\end{minipage}
   	\caption{مقایسه سود و هزینه الگوریتم‌های مختلف}
   	\label{fig:rev-cost1}
   \end{figure}

\subsection{درآمد و هزینه}


شکل‌های 
\ref{fig:rev-load1000} و \ref{fig:cost-load1000}
، به ترتیب‌، هزینه و درآمد الگوریتم‌های مختلف را مقایسه می‌کنند. محاسبات براساس معادلات 
\eqref {eq_rev} و \eqref {eq_cost}
 انجام می‌شود‌،  که  در آن‌ها مقادیر $ \ zeta_r = 1 $ و $ \ xi_r = 1 $ تنظیم شده. بنابراین‌، تأثیر منابع سرور همانند پهنای باند لینک است. از آنجا که \ourAlg\ در طی مدت مشابه \lr{VN} بیشتری تعبیه می‌کند‌، درآمد آن نیز بیشتر است. علیرغم درآمد بیشتر‌، الگوریتم پیشنهادی ما هزینه کمتری نسبت به روش \lr{First Fit } و \lr{Best Fit} دارد. از آنجا که \lr{GRC}  نسبت درآمد به هزینه را بهینه سازی می‌کند‌، هزینه کمتری نسبت به الگوریتم‌های \lr{First Fit } و \lr{Best Fit} دارد در حالی که درآمد تقریباً یکسانی را ارائه می دهد. \lr{NeuroViNE} به هزینه کمی دارد زیرا به طور معمول شبکه‌های مجازی کوچکتری را تعبیه می‌کند. در نتیجه‌، نمی‌تواند درآمد بالایی کسب کند. علاوه بر این‌، \ourAlg\ بالاترین نسبت درآمد به هزینه (= $ 1.87 $) را کسب می‌کند‌، در حالی که \lr{NeuroViNE}‌، \lr{GRC}‌، \lr{Best Fit} و \lr{First Fit} به ترتیب $1.57 $‌، $ 1.37 $‌،$ 1.25 $ و  $ 1.08 $ دارند. 
الگوریتم \lr{NeuroViNE} عملیات نگاشت را به یک مجموعه نسبتاً کوچک از سرورها محدود می‌کند، که منجر به درآمد ضعیف آن می‌شود. این مسئله در همه مکانیز‌های پیش پردازش  که سرورها را از بین می‌برند مشاهده می‌شود. از طرف دیگر‌، \ourAlg\ با معرفی مجموعه ای متنوع از نقاط شروع‌، الگوریتم نگاشت را به فضای کوچکتری راهنمایی می‌کند‌، اما به آن اجازه می‌دهد تا از اطلاعات همه سرورهای قابل دسترس استفاده کند. 
\subsection{بهره‌وری}
  	
  	\begin{figure}[t]
  		\centering
  		\begin{minipage}{.43\linewidth}
  			\centering
  			\resizebox{\linewidth}{!}{%
  				\input{plots/utilization/cpuBoxPlot}
  			}%
  			\caption{بهره‌وری \lr{CPU}}
  			\label{fig:bp-cpu-util}
  		\end{minipage}
  		\hfil
  		\begin{minipage}{.43\linewidth}
  			\centering
  			\resizebox{\linewidth}{!}{%
  				\input{plots/utilization/linkBoxPlot}
  			}%
  			\caption{بهره‌وری لینک‌ها}
  			\label{fig:bp-link-util}
  		\end{minipage}
  		\hfill
  		\caption{بیشترین بهره‌وری از سرور‌ها و لینک‌ها در طول شبیه‌سازی}
  		\label{fig:max-util}
  	\end{figure}
  
  نمودارهای جعبه ای در شکل
   \ref {fig:max-util} 
   توزیع حداکثر بهره‌وری در استفاده از سرورها و پیوندها را در کل شبیه سازی نشان می‌دهد. شکل
   \ref {fig:bp-cpu-util} 
   نشان می‌دهد که \ourAlg\ سرورهای فعال با گره‌های مجازی را بسته بندی می‌کند و فقط تعداد کمی از سرورها را کم استفاده می‌کند. علاوه بر این‌، \ourAlg\ استفاده از پهنای باند را برای حفظ اتصال در سطح متوسط ​​نگه می‌دارد (نگاه کنید به شکل
    \ref {fig:bp-link-util})
    ). الگوریتم‌های \lr{Best Fit} و \lr{NeuroViNE} سطح بالایی از تعادل بار در استفاده از \lr{CPU} را برای حفظ توانایی تأمین نیازهای آینده نشان می‌دهند. با این حال‌، آنها عملکرد کمتری در استفاده از پهنای باند از خود نشان می‌دهند‌، که می‌تواند ناشی از عدم در نظر گرفتن همزمان سرورها و شبکه باشد. به همین ترتیب‌، روش‌های \lr{GRC} و\lr{ First Fit} دارای بهره‌وری سرور و پیوند بالایی هستند که توانایی آنها را برای نگاشت شبکه‌های مجازی محدود می‌کند.
    
    \subsection{\gls{arrival rate} شبکه مجازی}
    
    \begin{figure}[t]
    	% \vspace{0.05in}
    	\centering
    	\begin{minipage}{.3\linewidth}
    		\centering
    		\resizebox{\linewidth}{!}{%
    			\input{plots/differentLoads/diffLoadsAR}
    		}%
    		\caption{نرخ پذیرش}
    		\label{fig:diff-load-ar}
    	\end{minipage}
    	\hfil
    	\begin{minipage}{.3\linewidth}
    		\centering
    		\resizebox{\linewidth}{!}{%
    			\input{plots/differentLoads/diffLoadsRevenue}
    		}%
    		\caption{درآمد}
    		\label{fig:diff-load-rev}
    	\end{minipage} 
    	\begin{minipage}{.3\linewidth}
    		\centering
    		\resizebox{\linewidth}{!}{%
    			\input{plots/differentLoads/diffLoadsCost}
    		}%
    		\caption{هزینه}
    		\label{fig:diff-load-cost}
    	\end{minipage}
    	\caption{مقایسه معیارها با نرخ‌های ورود شبکه‌ مجازی متفاوت}
    	\label{fig:diff-load}
    \end{figure}
    
    برای اندازه گیری بیشتر مقیاس پذیری و عملکرد‌، ما \ourAlg\ را با الگوریتم‌های دیگر تحت نرخ‌های مختلف ورود شبکه مجازی (یعنی بار‌های مختلف) مقایسه کردیم. به طور خاص‌، ما نرخ‌های مختلف ورود را بین $ 1.2 $ و $ 4.4 $ بررسی کردیم  و نتایج را در شکل  \ref{fig:diff-load}  ارائه دادیم.
    از شکل \ref{fig:diff-load-ar} مشهود است که با افزایش نرخ ورود‌، میزان پذیرش درخواست‌ها کاهش می‌یابد. با این وجود‌، میزان قبولی حاصل از \ourAlg\ در مقایسه با الگوریتم‌های دیگر به طور مداوم بالاتر است. مشاهده می‌کنیم که \lr{NeuroViNE} مقیاس پذیری بهتری از خود نشان می‌دهد‌، اما عملکرد آن هنوز حدود 18 درصد کمتر از روش ما است. \lr{NeuroViNE} درآمد و هزینه کمی دارد که نشان می‌دهد نسبت پذیرش بالاتر آن از توانایی آن در جاسازی \lr{VN}‌های کوچکتر ناشی می‌شود که درآمد کمتری دارند.
     اگرچه نرخ پذیرش درخواست‌ها کاهش می‌یابد‌، اما \ourAlg\ موفق می‌شود تعداد شبکه‌های مجازی تعبیه شده را افزایش دهد که منجر به درآمد و هزینه بالاتر می‌شود (به شکل \ref{fig:diff-load-rev} و \ref{fig:diff-load-cost} نگاه کنید). با این حال‌، الگوریتم‌های دیگر در محدودیت $ 2 $ شبکه مجازی در هر واحد از زمان هستند‌، و در نتیجه درآمد و هزینه آنها تغییر قابل توجهی نمی‌کند.
     
      \subsection{تقاضای منبع پیوند مجازی}
      
      \begin{figure*}[t]
      	\centering
      	% \vspace{0.03in}
      	\begin{minipage}{.29\linewidth}
      		\centering
      		\resizebox{\linewidth}{!}{%
      			\input{plots/differentLinks/diffLinksAR}
      		}%
      		\caption{نرخ پذیرش}
      		\label{fig:diff-link-ar}
      	\end{minipage}
      	\hfil
      	\begin{minipage}{.29\linewidth}
      		\centering
      		\resizebox{\linewidth}{!}{%
      			\input{plots/differentLinks/diffLinksRevenue}
      		}%
      		\caption{درآمد}
      		\label{fig:diff-link-rev}
      	\end{minipage}
      	\hfil
      	\begin{minipage}{.29\linewidth}
      		\centering
      		\resizebox{\linewidth}{!}{%
      			\input{plots/differentLinks/diffLinksCost}
      		}%
      		\caption{هزینه}
      		\label{fig:diff-link-cost}
      	\end{minipage}
      	\caption{تاثیر تقاضاهای مختلف پیوند‌ مجازی}
      	\label{fig:diff-link}
      \end{figure*}
  
  در این آزمایش‌، ما تأثیر تقاضاهای  مختلف پیوند مجازی را بر عملکرد الگوریتم‌های مختلف بررسی می‌کنیم. بنابراین‌، ما واریانس درخواست‌های پیوند مجازی را که توزیعی نرمال با میانگین 4  است‌، از 10  به 90  تغییر می‌دهیم و نتایج را در شکل \ref{fig:diff-link} ارائه می‌دهیم.
  مشاهده می‌کنیم که \ourAlg\ قادر به نگاشت پیوندهای مجازی گسترده تر در سرورهای فیزیکی است و بنابراین نسبت پذیرش آن تغییر نمی‌کند که منجر به افزایش درآمد می‌شود (نگاه کنید به شکل \ref{fig:diff-link-rev} ). این روند برای الگوریتم های \lr{NeuroViNE}‌،\lr{ Best Fit} و \lr{GRC} نیز اتفاق می‌افتد‌، که نشان می‌دهد آنها نیز به اهمیت نگاشت پیوندهای مجازی بزرگ در داخل سرورهای فیزیکی آگاه هستند. با این حال‌، الگوریتم \lr{Firs Fit} از تقاضای پیوند مجازی بالا رنج می‌برد‌، که به طور قابل توجهی نسبت پذیرش آن را کاهش می‌دهد.
  \subsection{منابع چند‌بعدی}
  
      \begin{figure*}[t]
      	\centering
      	\begin{minipage}{.3\linewidth}
      		\centering
      		\resizebox{\linewidth}{!}{%
      			\input{plots/acceptanceRatios/with_extra_features_load1000}
      		}%
      		\caption{نرخ پذیرش}
      		\label{fig:ar-load1000-with-gpu}
      	\end{minipage}
      	\hfil
      	\begin{minipage}{.3\linewidth}
      		\centering
      		\resizebox{\linewidth}{!}{%
      			\input{plots/extraFeaturesCostRevenue/load1000}
      		}%
      		\caption{هزینه و درآمد}
      		\label{fig:cost-rev-load1000-with-gpu}
      	\end{minipage}
      	\hfil
      	\begin{minipage}{.3\linewidth}
      		\centering
      		\resizebox{\linewidth}{!}{%
      			\input{plots/utilization/load1000WithExtraFeatures/uitls}
      		}%
      		\caption{میانگین بهره‌وری}
      		\label{fig:avg-util-with-gpu}
      	\end{minipage}
      	\caption{نتایج شبیه‌سازی الگوریتم‌ها با منابع اضافی (\lr{GPU} و \lr{RAM})}
      	\label{fig:multi_feature}
      \end{figure*}
  
  الگوریتم \ourAlg\ عملیات نگاشت شبکه مجازی را در یک تنظیمات با منابع چندبعدی در سرور‌ها پشتیبانی می‌کند. در این آزمایش‌، ضمن در نظر گرفتن دو منبع اضافی ( \lr{GPU} و \lr{RAM})‌، عملکرد \ourAlg\ را بررسی می‌کنیم. از آنجا که \lr{GRC} و \lr{NeuroViNE} از تخصیص منابع چند بعدی پشتیبانی نمی‌کنند‌، در اینجا آنها را حذف می‌کنیم. ما  \lr{First Fit} را تغییر دادیم تا اولین سرور فیزیکی را با ظرفیت کافی برای هر نوع منبع انتخاب کند. همچنین  \lr{Best Fit} برای انتخاب سرور طوری اصلاح شده است که مجموع ظرفیت منابع باقیمانده آن حداکثر باشد. شکل 
  \ref {fig:multi_feature} 
  نشان می‌دهد که \ourAlg\ از الگوریتم های دیگر در هر معیاری بهتر عمل می‌کند. به طور خاص‌، \ourAlg\ حدود $70\%$  شبکه‌های مجازی بیشتری می‌پذیرد و درآمد را تا حدود $ 3 $ برابر بهبود می‌بخشد. با توجه به شکل
  \ref {fig:cost-rev-load1000-with-gpu}
   می‌توان گفت که  \ourAlg\ تعداد قابل توجهی پیوند مجازی بزرگ را در داخل سرورهای فیزیکی تعبیه کرده و بنابراین هزینه نگاشت آنها در شبکه فیزیکی را پرداخت نمی‌کند. در نتیجه‌، \ourAlg\ هزینه‌ی کمتر نسبت به درآمد را عملی می‌کند‌، که یک حالت ایده آل است. اما \lr{Best Fit } و \lr{First Fit } بسیاری از پیوندهای مجازی را به چندین پیوند فیزیکی ترسیم می‌کنند و بنابراین هزینه آنها بیش از درآمد آنها است. این نتیجه‌گیری بیشتر در شکل
    \ref{fig:avg-util-with-gpu} 
    تأیید می‌شود که در آن \ourAlg\ در مقایسه با سایر روش‌ها‌، بهره‌وری بیشتر از منابع و بهره‌وری کمتر از شبکه را نشان می‌دهد. 
      
     
    
