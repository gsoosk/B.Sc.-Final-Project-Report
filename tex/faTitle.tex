% !TeX root=../main.tex
% در این فایل، عنوان پایان‌نامه، مشخصات خود، متن تقدیمی‌، ستایش، سپاس‌گزاری و چکیده پایان‌نامه را به فارسی، وارد کنید.
% توجه داشته باشید که جدول حاوی مشخصات پروژه/پایان‌نامه/رساله و همچنین، مشخصات داخل آن، به طور خودکار، درج می‌شود.
%%%%%%%%%%%%%%%%%%%%%%%%%%%%%%%%%%%%
% دانشگاه خود را وارد کنید
\university{دانشگاه تهران}
% پردیس دانشگاهی خود را اگر نیاز است وارد کنید (مثال: فنی، علوم پایه، علوم انسانی و ...)
\college{پردیس دانشکده‌های فنی}
% دانشکده، آموزشکده و یا پژوهشکده  خود را وارد کنید
\faculty{دانشکده‌ی برق و کامپیوتر}
% گروه آموزشی خود را وارد کنید (در صورت نیاز)
\department{گروه نرم‌افزار}
% رشته تحصیلی خود را وارد کنید
\subject{مهندسی کامپیوتر}
% گرایش خود را وارد کنید
\field{نرم‌افزار}
% عنوان پایان‌نامه را وارد کنید
\title{
سرعت بخشیدن به الگوریتم نگاشت شبکه مجازی با استفاده از شبکه‌های عصبی گرافی
}
% نام استاد(ان) راهنما را وارد کنید
\firstsupervisor{دکتر  احمد خونساری}
\firstsupervisorrank{دانشیار}
%\secondsupervisor{دکتر راهنمای دوم}
%\secondsupervisorrank{استادیار}
% نام استاد(دان) مشاور را وارد کنید. چنانچه استاد مشاور ندارید، دستورات پایین را غیرفعال کنید.
%\firstadvisor{دکتر مشاور اول}
%\firstadvisorrank{استادیار}
%\secondadvisor{دکتر مشاور دوم}
% نام داوران داخلی و خارجی خود را وارد نمایید.
\internaljudge{دکتر بهنام بهرک}
\internaljudgerank{استادیار}
%\externaljudge{دکتر داور خارجی}
%\externaljudgerank{دانشیار}
%\externaljudgeuniversity{دانشگاه داور خارجی}
% نام نماینده کمیته تحصیلات تکمیلی در دانشکده \ گروه
%\graduatedeputy{دکتر نماینده}
%\graduatedeputyrank{دانشیار}
% نام دانشجو را وارد کنید
\name{فرزاد}
% نام خانوادگی دانشجو را وارد کنید
\surname{حبیبی}
% شماره دانشجویی دانشجو را وارد کنید
\studentID{810195383}
% تاریخ پایان‌نامه را وارد کنید
\thesisdate{بهمن ۱۳۹۹}
% به صورت پیش‌فرض برای پایان‌نامه‌های کارشناسی تا دکترا به ترتیب از عبارات «پروژه»، «پایان‌نامه» و «رساله» استفاده می‌شود؛ اگر  نمی‌پسندید هر عنوانی را که مایلید در دستور زیر قرار داده و آنرا از حالت توضیح خارج کنید.
%\projectLabel{پایان‌نامه}

% به صورت پیش‌فرض برای عناوین مقاطع تحصیلی کارشناسی تا دکترا به ترتیب از عبارت «کارشناسی»، «کارشناسی ارشد» و «دکتری» استفاده می‌شود؛ اگر نمی‌پسندید هر عنوانی را که مایلید در دستور زیر قرار داده و آنرا از حالت توضیح خارج کنید.
%\degree{}
%%%%%%%%%%%%%%%%%%%%%%%%%%%%%%%%%%%%%%%%%%%%%%%%%%%%
%% پایان‌نامه خود را تقدیم کنید! %%
%\dedication
%{
%{\Large تقدیم به:}\\
%\begin{flushleft}{
%	\huge
%	همسر و برادران \\
%	\vspace{7mm}
%	و\\
%	\vspace{7mm}
%	پدر و مادرم
%}
%\end{flushleft}
%}
%% متن قدردانی %%
%% ترجیحا با توجه به ذوق و سلیقه خود متن قدردانی را تغییر دهید.
\acknowledgement{
سپاس خداوندگار حکیم را که با لطف بی‌کران خود، آدمی را به زیور عقل آراست.

در آغاز وظیفه‌  خود  می‌دانم از زحمات بی‌دریغ اساتید  راهنمای خود،  جناب آقای دکتر خونساری، صمیمانه تشکر و  قدردانی کنم که در طول انجام این پایان‌نامه با نهایت صبوری همواره راهنما و مشوق من بودند و قطعاً بدون راهنمایی‌های ارزنده‌ ایشان، این مجموعه به انجام نمی‌رسید.

از جناب آقای دکتر دولتی که  زحمت مشاوره‌ و کمک در این پایان‌نامه را تقبل فرمودند کمال امتنان را دارم.

%از همکاری و مساعدت‌های دکتر ... مسئول تحصیلات تکمیلی و سایر کارکنان دانشکده بویژه سرکار خانم ... کمال تشکر را دارم.

با سپاس بی‌دریغ خدمت دوستان گران‌مایه‌ام که با همفکری مرا صمیمانه و مشفقانه یاری داده‌اند.

و در پایان، بوسه می‌زنم بر دستان خداوندگاران مهر و مهربانی، پدر و مادر عزیزم و بعد از خدا، ستایش می‌کنم وجود مقدس‌شان را و تشکر می‌کنم از خانواده عزیزم به پاس عاطفه سرشار و گرمای امیدبخش وجودشان، که بهترین پشتیبان من بودند.
}
%%%%%%%%%%%%%%%%%%%%%%%%%%%%%%%%%%%%
%چکیده پایان‌نامه را وارد کنید
\fa-abstract{
مجازی‌سازی شبکه یکی از برجسته‌ترین فناوری‌ها در شبکه‌های کامپیوتری می‌باشد که اجازه‌ پیاده‌سازی نسل جدید برنامه‌های مبنی بر شبکه را در یک زیرساخت مشترک میسر می‌سازد. این تکنولوژی اجازه ورود هم‌زمان و مشترک چند شبکه‌ مجازی را در یک بستر فیزیکی، که به عنوان شبکه‌بستر شناخته می‌شود را ارائه می‌دهد و جزئی مهم در شبکه‌های ابری و چارچوب کاری «\gls{Infrastructure as a Service}»
% \LTRfootnote{Infrastructure as a Service} 
می‌باشد. تخصیص منبع در این تکنولوژی با استفاده‌ از  الگوریتم‌هایی تحت عنوان نگاشت شبکه مجازی 
\lr{(VNE)}
 انجام می‌شود که وظیفه‌ی نگاشت منابع مجازی به توپولوژی شبکه‌بستر فیزیکی را به صورت استفاده حداکثری از منابع فیزیکی، دارد. در این پروژه ما قصد داریم داریم با بهره‌گیری از شبکه‌های عصبی گرافی با انجام پیش‌پردازش بر روی شبکه‌های مجازی درخواستی و شبکه فیزیکی موجود، توانایی انجام این الگوریتم‌ها را از منظر مقیاس‌پذیری ، بهره‌وری و درآمد بهبود ببخشیم. 
 کار‌های قبلی در این مسئله، به صورت کلی بر کارایی و  استفاده حداکثری از منابع تمرکز داشته‌‌اند و مقایس‌پذیری را به عنوان یک هدف اصلی در نظر نگرفته‌اند. بنابراین، افزایش روزافزون درخواست‌ها و اندازه‌ آن‌ها، این راه‌حل‌ها را کمتر کاربردی در محیط عملی می‌کند. هدف ما در این پروژه افزایش عملکرد و بازدهی الگوریتم‌های موجود نسبت به کار‌های پیشین و در عین‌حال \gls{scalability} قابل قبول و کاربردی با بهره‌گیری از قابلیت موازی‌سازی شبکه‌های عصبی می‌باشد. همین‌طور این راه‌حل می‌تواند به عنوان یک بلوک‌سازنده در نسل جدید شبکه‌ و شبکه‌های خودکار مورد استفاده قرار بگیرد. 
 آزمایش های ما با استفاده از شبیه سازی نشان می دهد که موازی سازی  زمان اجرای آن را با ضریب  ۸ کاهش می‌دهد. همچنین این الگوریتم در مقایسه با الگوریتم های شبیه‌سازی شده دیگر ، نسبت درآمد به هزینه را با حدود ۱۸ درصد بهبود می‌بخشد.
}
% کلمات کلیدی پایان‌نامه را وارد کنید
\keywords{
نگاشت شبکه مجازی - شبکه‌های عصبی گرافی - مجازی سازی - شبکه مجازی
}
% انتهای وارد کردن فیلد‌ها
%%%%%%%%%%%%%%%%%%%%%%%%%%%%%%%%%%%%%%%%%%%%%%%%%%%%%%
